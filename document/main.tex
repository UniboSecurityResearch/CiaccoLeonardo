%%%%%%%%%%%%%%%%%%%%%%%%%%%%%%%%%%%%%%%%
%12pt: grandezza carattere
%a4paper: formato a4
%openright: apre i capitoli a destra
%twoside: serve per fare un documento fronteretro
%report: stile tesi (oppure book)
\documentclass[12pt,a4paper,openright,twoside]{report}

\usepackage[italian]{babel}
\usepackage[latin1]{inputenc}
\usepackage{fancyhdr}
\usepackage{indentfirst}
\usepackage{graphicx}
\usepackage{newlfont}
\usepackage{textcomp}
%librerie matematiche
\usepackage{amssymb}
\usepackage{amsmath}
\usepackage{latexsym}
\usepackage{amsthm}
\usepackage{listings}
\usepackage{chngcntr}
\usepackage[chapter]{minted}


\oddsidemargin=30pt \evensidemargin=20pt%impostano i margini
\hyphenation{sil-la-ba-zio-ne pa-ren-te-si}

\usepackage[square, numbers, comma, sort&compress]{natbib} %Bibliografia
\usepackage{hyperref}
\usepackage{float}

\usepackage[pass]{geometry}

%comandi per l'impostazione della pagina, vedi il manuale della libreria fancyhdr per ulteriori delucidazioni
\pagestyle{fancy}\addtolength{\headwidth}{20pt}
\renewcommand{\chaptermark}[1]{\markboth{\thechapter.\ #1}{}}
\renewcommand{\sectionmark}[1]{\markright{\thesection \ #1}{}}
\rhead[\fancyplain{}{\bfseries\leftmark}]{\fancyplain{}{\bfseries\thepage}}
\cfoot{}

\linespread{1.3} %comando per impostare l'interlinea

\renewcommand{\listingscaption}{Codice}
\renewcommand{\listoflistingscaption}{Elenco dei Codici}

\begin{document}
%%%%%%%%%%%%%%%%%%%%%%%%%%%%%%%%%%%%%%%%
% FRONTESPIZIO
\newgeometry{hmarginratio=1:1}
\begin{titlepage}
\begin{center}
\includegraphics[width=2.56in]{figures/logo/logo_unibo.png}\\
% {{\Large{\textsc{Alma Mater Studiorum $\cdot$ Universit\`a di
% Bologna}}}} \rule[0.1cm]{14.7cm}{0.1mm}
% \rule[0.5cm]{14.7cm}{0.6mm}
\vspace{5mm}
{\small{\bf SCUOLA DI INGEGNERIA E ARCHITETTURA\\
\vspace{2mm}
Dipartimento di Informatica -- Scienza e Ingegneria\\
\vspace{2mm}
Corso di Laurea in Ingegneria Informatica }}
%se Laurea Magistrale scrivere "Corso di Laurea Magistrale in Ingegneria Informatica"
\end{center}
\vspace{3,8mm}
\begin{center}
{\LARGE{\bf Cybersecurity e finanza:\\
\vspace{3,5mm}
analisi e studio delle conseguenze\\
\vspace{3,5mm}
finanziare degli attacchi informatici\\
\vspace{3,5mm}
sulle aziende\\
\vspace{3,5mm}}}
  
\end{center}
\vspace{9,5mm}
\par
\noindent
\begin{minipage}[t]{0.47\textwidth}
{\normalsize{\bf Relatore:\\
Prof. Dr. Marco Prandini\\}}
{\normalsize{\bf Correlatore:\\
Dr. Alessandro Vannini}}
\end{minipage}
\hfill
\begin{minipage}[t]{0.47\textwidth}\raggedleft
{\normalsize{\bf Presentata da:\\
Leonardo Ciacco}}
\end{minipage}
\vspace{15mm} % da aumentare a 10, 20 o 30 se non si inseriscono i correlatori
\begin{center}
{\normalsize{\bf Appello unico\\%inserire il numero dell'appello in cui ci si laurea CAPIRE
Anno Accademico 2024-2025}}%inserire l'anno accademico a cui si è iscritti
\end{center}
\end{titlepage}

\restoregeometry


%%%%%%%%%%%%%%%%%%%%%%%%%%%%%%%%%%%%%%%%
% ABSTRACT 
\clearpage{\pagestyle{empty}\cleardoublepage}
\pagenumbering{roman}
\renewcommand{\abstractname}{Abstract}
\phantomsection
\addcontentsline{toc}{chapter}{Abstract}
\begin{abstract}
    Questo \`e l'abstract: un riassunto dell'introduzione di massimo 300 parole. Da scrivere alla fine.
\end{abstract}


%%%%%%%%%%%%%%%%%%%%%%%%%%%%%%%%%%%%%%%%
% RINGRAZIAMENTI 
\clearpage{\pagestyle{empty}\cleardoublepage}
\chapter*{Ringraziamenti}
\phantomsection
\addcontentsline{toc}{chapter}{Ringraziamenti}
Qui possiamo ringraziare il mondo intero!!!!!!!!!!\\
Ovviamente solo se uno vuole, non \`e obbligatorio.

\clearpage{\pagestyle{empty}\cleardoublepage}



%%%%%%%%%%%%%%%%%%%%%%%%%%%%%%%%%%%%%%%%
% INDICE 
\tableofcontents %crea l'indice
%imposta l'intestazione di pagina
\rhead[\fancyplain{}{\bfseries\leftmark}]{\fancyplain{}{\bfseries\thepage}}
\lhead[\fancyplain{}{\bfseries\thepage}]{\fancyplain{}{\bfseries
Indice}}



%%%%%%%%%%%%%%%%%%%%%%%%%%%%%%%%%%%%%%%%
% ELENCO DELLE FIGURE
\clearpage{\pagestyle{empty}\cleardoublepage}
\renewcommand{\listfigurename}{Elenco delle Figure}
\phantomsection
\addcontentsline{toc}{chapter}{Elenco delle Figure}
\listoffigures %crea l'elenco delle figure



%%%%%%%%%%%%%%%%%%%%%%%%%%%%%%%%%%%%%%%%
% ELENCO DELLE TABELLE
\clearpage{\pagestyle{empty}\cleardoublepage}
\renewcommand{\listtablename}{Elenco delle Tabelle}
\phantomsection
\addcontentsline{toc}{chapter}{Elenco delle Tabelle}
\listoftables %crea l'elenco delle tabelle



%%%%%%%%%%%%%%%%%%%%%%%%%%%%%%%%%%%%%%%%
% INTRODUZIONE
\clearpage{\pagestyle{empty}\cleardoublepage}
\chapter{Introduzione} 
\lhead[\fancyplain{}{\bfseries\thepage}]{\fancyplain{}{\bfseries\rightmark}}
\pagenumbering{arabic} %mette i numeri arabi
Questa \`e l'introduzione: da scrivere alla fine, un breve riassunto su cosa si andr\`a ad affrontare nella tesi.
Le linee guida per la stesura della Tesi di Laurea sono al seguente link \href{https://ulis.se/thesis/}{https://ulis.se/thesis/}.

\section{Prima Sezione} %crea la sezione
Questa \`e la prima sezione.


Ora vediamo un elenco numerato: %crea un elenco numerato
\begin{enumerate}
\item primo oggetto
\item secondo oggetto
\item terzo oggetto
\item quarto oggetto
\end{enumerate}

\begin{figure}[h] %crea l'ambiente figura; [h] sta per here, cioè la figura va qui
\begin{center} %centra nel mezzo della pagina la figura \includegraphics[width=5cm]{figura.eps} inserisce una figura larga 5cm se si vuole usare va scommentata

%inserisce la legenda ed etichetta la figura con \label{fig:prima}
\caption[legenda elenco figure]{legenda sotto la figura}\label{fig:prima}
\end{center}
\end{figure}


\section{Seconda Sezione}
Ora vediamo un elenco puntato:
\begin{itemize} %crea un elenco puntato
\item primo oggetto
\item secondo oggetto
\end{itemize}


\section{Altra Sezione}
Vediamo un elenco descrittivo:

\begin{description} %crea un elenco descrittivo
  \item[OGGETTO1] prima descrizione;
  \item[OGGETTO2] seconda descrizione;
  \item[OGGETTO3] terza descrizione.
\end{description}


\subsection{Altra SottoSezione}
%crea una sottosottosezione

\subsubsection{SottoSottoSezione}Questa sottosottosezione non viene
numerata, ma \`e solo scritta in grassetto.


\section{Altra Sezione} %crea una sottosezione
Vediamo la creazione di una tabella; la tabella \ref{tab:uno}
(richiamo il nome della tabella utilizzando la label che ho messo sotto):
la facciamo di tre righe e tre colonne, la prima colonna
``incolonnata'' a destra (r) e le altre centrate (c):\\
\begin{table}[h] %ambiente tabella (serve per avere la legenda)
\begin{center} %centra nella pagina la tabella
\begin{tabular}{r|c|c} %tre colonne con righe verticali prodotte con |
\hline \hline                           %inserisce due righe orizzontali
$(1,1)$ & $(1,2)$ & $(1,3)$\\ %& separa le colonne e con
\hline                                  %inserisce una riga orizzontale
$(2,1)$ & $(2,2)$ & $(2,3)$\\ %  \\ va a capo
\hline                                  %inserisce una riga orizzontale
$(3,1)$ & $(3,2)$ & $(3,3)$\\
\hline \hline                           %inserisce due righe orizzontali
\end{tabular}
\caption[legenda elenco tabelle]{legenda tabella}\label{tab:uno}
\end{center}
\end{table}


\section{Altra Sezione}\label{sec:prova}%posso mettere le label anche alle section
In questa sezione voglio fare un riferimento alla bibliografia: questo \`e il mio riferimento 
prova cit \cite{enisa_threat_landscape}

\subsection{Listati dei programmi}
\subsubsection{Primo Listato}
\begin{verbatim}
        In questo ambiente     posso scrivere      come voglio,
lasciare gli spazi che voglio e non % commentare quando voglio
e ci sarà scritto tutto.
Quando lo uso è meglio che disattivi il Wrap del WinEdt
\end{verbatim}
\clearpage{\pagestyle{empty}\cleardoublepage}



%%%%%%%%%%%%%%%%%%%%%%%%%%%%%%%%%%%%%%%%
% STATO DELL'ARTE
\chapter{Analisi dello stato dell'arte e del contesto}
Il presente capitolo mira a fornire le informazioni fondamentali, necessarie a comprendere i futuri sviluppi. A questo scopo sono definiti alcuni concetti chiave in ambito di sicurezza informatica:
\begin{itemize}
    \item rischio
    \item sicurezza
    \item minaccia
    \item vulnerabilit\`a
    \item attacco
\end{itemize}
Con \textbf{rischio} si intende la possibilit\`a che azioni umane o eventi  abbiano un impatto su ci\`o a cui \`e attribuito valore. \`E quindi corretto affermare che si tratta di una grandezza, e come tale \`e quantificabile, \`e infatti direttamente proporzionale alla probabilit\`a di avvenimento e all'impatto che avrebbe sui beni colpiti. La valutazione del rischio, mirata alla sua gestione e controllo, parte dalla stima quanto pi\`u accurata di queste due variabili.\\
\\
La \textbf{sicurezza} \`e il processo di contenimento del rischio che consiste essenzialmente nella cura di tre propriet\`a fondamentali: riservatezza (inaccessibilit\`a delle informazioni a chi non ha il permesso di consultarle), integrit\`a (garanzia di correttezza delle informazioni) e disponibilit\`a(possibilit\`a di accesso e utilizzo delle informazioni e i servizi offerti).\\
\\
La \textbf{minaccia} consiste nella potenziale compromissione di una o pi\`u delle propriet\`a sopracitate, e, da concetto astratto, si concretizza in un \textbf{attacco}: azione di sfruttamnento (\textit{exploit}) delle vulnerabilit\`a di un sistema\\

\section{Tipologie di attacco}
ENISA (Agenzia dell'Unione europea per la cibersicurezza) raggruppa gli attacchi informatici in otto categorie principali.\cite{enisa_threat_landscape}
\subsection{Ransomware}
Definito come un tipo d attacco in cui gli attori coinvolti prendono il controllo di un bene del bersaglio e chiedono un riscatto (in inglese \textit{ransom}) in cambio della restituzione. \`E un tipo di attacco fatto generalmente per scopi economici, ma, essendo anche uno dei pi\`u diffusi evolve rapidamente, sia dal punto di vista delle tecniche di estorsione che degli obiettivi.\\

\subsection{Malware}
Si tratta di un termine ombrello, descrive tutti i software o firmware con lo scopo di eseguire un processo non autorizzato sulla macchina ospitante il cui risultato \`e la compromissione di una delle tre propiet\`a della sicurezza.\\
\subsection{Ingegneria sociale}
Gli attacchi tramite ingegneria sociale si distinguono per lo sfruttamento della vulnerabilit\`a derivata dall'errore umano. Consinstono nel mettere in atto tecniche di manipolazione per indurre chi coinvolto nella gestione dei beni bersaglio a rilasciare informazioni sensibili, concedere permessi 
apparentemente innocui, visitare siti, aprire file, documenti o e-mails con contenuto malevolo. Un esempio pu\`o essere il \textit{phishing}, attacco in cui si cerca di indurre la vittima ad aprire un link ad un sito che scarica un malware.\\
L'ingegneria sociale \`e spesso protagonista delle fasi iniziali di un attacco, ma questo non esclude che possa essere impiegata in stadi pi\`u avanzati. Ad esempio, \`e chiamata BEC (\textit{Business E-mail Compromise}) l'intrusione tramite vecchie credenziali aziendali che le sfrutta, fingendosi parte dell'organizzazione, per ottenere dagli ex colleghi informazioni, denaro, o risorse altrimenti inaccessibili.\cite{IBM_BEC}\\

\subsection{Violazioni di dati}
Tecnicamente vi ci si riferisce con i termini inglesi \textit{data breach} o \textit{data leak}.  
Da definizione del GDPR, per violazione dei dati personaliintendiamo la "violazione di sicurezza che comporta accidentalmente o in modo illecito la distruzione, la perdita, la modifica, la divulgazione non autorizzata o l'accesso ai dati personali trasmessi, conservati o comunque trattati" (articolo 4.12 GDPR). \\
La differenza risiede nell'intenzionalità: il \textit{breach} \`e un attacco con l'esplicito obiettivo di ottenere o diffondere dati sensibili o protetti, il \textit{leak} è originato da un errore che porta ad una indesiderata perdita o messa a disposizione di dati sensibili. \\

\subsection{Minacce alla disponibilit\`a: DoS}
Per definizione, un DDoS (\textit{Distriubuted Denial of Service}), si verifica come risultato di un attacco che mira a rendere inaccessibili servizi o risorse tramite, per esempio, uno sproporzioato volume di richieste che sovraccarica i componenti della struttura attaccata.
In questo caso quindi l'obiettivo risiede nella semplice interruzione dei servizi erogati.\\

\subsection{Minacce alla disponibilit\`a: minacce ad Internet}
Alla base della disponibilit\`a dei servizi nella societ\`a dell'informazione vi \`e l'infraastruttura di internet, che se subisce guasti distribuiti pu\`o comprometterne quasi completamente l'erogazione. Sebbene sia un'ipotesi remota,  blackout, disastri naturali, interruzione forzata da parte del governo o attacchi su larga scala rappresentano una minaccia concreta che non pu\`o essere sottovalutata.\\

\subsection{Manipolazione delle informazioni}
Si chiama \textit{Foreign Information Manipulation and Interference}, abbreviato in FIMI, l'insieme di strategie di manipolazione dell'opinione pubblica da parte di enti, governativi e non, volte ad inserirsi nel dibattito in maniera massiccia e organizzata. Ci\`o mina i processi decisionali democratici tramite campagne di disinformazione e narrazioni precostruite.\\
\subsection{Attacchi alla catena di fornituta}
Pi\`u comunemente  chiamati \textit{supply chain attacks}, sono attacchi che sfruttano la fiducia riposta nei servizi offerti da partner commerciali o fornitori, colpendo l'obiettivo indirettamente tramite la compromissione di un fornitore legittimo.\\
Pi\`u formalmente, un attacco \`e considerato in parte di natura \textit{supply chain} se consiste nella combinazione di almeno due attacchi coordinati. Perch\'e sia identificato come tale, devono infatti essere bersaglio sia l'obiettivo intermedio, fornitore, che l'obiettivo finale, naturalmente vulnerabile nei confronti chi valuta fonte sicura. \\
%SolarWinds
%was one of the first revelations of this kind of attack and showed the potential impact of supply chain attacks. It was observed that threat actors are continuing to feed on this source to conduct their operations and gain a
%foothold within organisations, to benefit from the widespread impact and large victim base of such attacks



\section{Contesto economico}
Con l'esponenziale crescita ed espansione dei mezzi di informazione, crescono anche i mercati illeciti, precisamente il costo del cybercrimine era di \$3 trilioni nel 2015 e, si stima, toccher\`a i \$10.5 trilioni durante il 2025\cite{cybercrime_magazine}. Se fosse il PIL di uno stato sarebbe sicuramente tr i primi cinque al mondo.\\
In un paorama di questo tipo le aziende quotate in borsa, soprattutto le pi\`u in risalto come quelle in indici come S\&P500, sono un bersaglio estremamente appetibile per chiunque abbia i mezzi e l'intenzione di danneggiarle in quanto:
\begin{itemize}
  \setlength{\itemsep}{0pt}
  \setlength{\parskip}{0pt}       
  \renewcommand{\labelitemi}{\textbf{--}} 
  \item sono in possesso di grandi quantit\`a di \textbf{dati sensibili}: questo sopratutto se si tratta di grandi aziende o se specializzate nel settore, ma chiaramente, se l'obiettivo \`e il furto di informazioni, un'alta concentrazione delle stesse agisce da incentivo;
  \item sono facilmente \textbf{valutabili}: per eseguire un attacco \textit{ransomware} occorre anche commisurare la richiesta di riscatto al valore del bene per l'azienda e alla sua disponibilit\`a economica, i bilanci pubblici offrono qundi una fonte di informazioni  preziosa;
  \item se attaccate, hanno \textbf{reazioni} spesso \textbf{prevedibili}: secondo un rapporto di Accenture del 2010 \cite{accenture2010}, le aziende che subiscono un attacco sperimentano un calo di prezzo delle azioni che si aggira solitmente intorno al 5\%, rivelando come l'attacco potrebbe essere sfruttato ulteriormente come strumento di controllo illecito del mercato per trarre  ulteriori profitti collaterali.
\end{itemize}
Solo nel 2023, il 21\% selle aziende in S\&P500 hanno subito un \textit{breach}\cite{SecurityScorecard_SP500}.\\
Inoltre, i danni connessi agli attacchi sono ugualmente preoccupanti, tra questi:
\begin{itemize}
  \setlength{\itemsep}{0pt}   
  \setlength{\parskip}{0pt}       
  \renewcommand{\labelitemi}{\textbf{--}}  
  \item ripercussioni legali;
  \item perdita di fiducia di clienti ed investitori.
\end{itemize}

\section{Metodologia}
Lo studio si propone di analizzare 15 casi di attacchi informatici, con focus sulle ripercussioni che questi hanno avuto sull'azienda.\\
Gli attacchi sono di vario tipo, e sono avvenuti tra il 2011 e il 2024 l'analisi \`e condotta con l'obiettivo di individuare un trend e in generale valutare la minacci crescente del cybercrimine.\\  
\clearpage{\pagestyle{empty}\cleardoublepage}



%%%%%%%%%%%%%%%%%%%%%%%%%%%%%%%%%%%%%%%%
% Attacchi e conseguenze finanziarie
\chapter{Attacchi e conseguenze finanziarie}
Nota: la maggior parte dei grafici proviene da \href{https://www.macrotrends.net/}{macrotrends}.
%\section{Epsilon data breach (2011)}
%\subsection{Descrizione}

%\subsection{Impatto diretto}

%\subsection{Impatto indiretto}

%\subsection{Note}
\section{Sony Playstation Network (2011)}
\subsection{Descrizione}
Sony dichiar\`o che tra il 17 e il 19 aprile 2011 avvene un importante \textit{data breach} , a seguito di cui disabilit\`o per una settimana il servizio online Playstation Network. L'attacco port\`o al furto di dati sensibili riguardanti 77 milioi di account\cite{Sony_PNT_guardian}\cite{Sony_pnt}.\\
\subsection{Impatto diretto}
Sony sub\`i diverse conseguenze legali, che unite agli obblighi di miglioramento della sicurezza, si stima ebbero un costo complessivo di \$171 milioni.\\
\subsection{Impatto indiretto}
Come si pu\`ovedere dalla figura ... e nelle fonti \cite{Sony_pnt} la compagnia sub\`i, nei primi giorni, un calo del valore delle azioni ben superiore al 3\%.\\

\begin{figure}[H] 
\begin{center} 
\includegraphics[width=10cm]{figures/sony_2011_shortTerm.png} 
\caption[Grafico Sony PSN short]{Impatto a breve termine Sony Playstation Network Outage}\label{fig:pnt1}
\end{center}
\end{figure}

A seguito inizia un periodo di contrazione che, possiamo osservare, porta i valori delle azioni da pi\`u di \$5 nel periodo antecedente alla crisi, fino a scendere sotto i \$2 nel 2013 (adjusted). \\Associare la parabola discendente esclusivamente all'attacco \`e forse esagerato, ci\`o non toglie che si tratti sicuramente di un fattore rilenvante.\\ 

\begin{figure}[H] 
\begin{center} 
\includegraphics[width=10cm]{figures/sony_2011_longTerm.png} 
\caption[Grafico Sony PSN long]{Impatto a lungo termine Sony Playstation Network Outage}\label{fig:pnt2}
\end{center}
\end{figure}


\subsection{Note}
Le comunicazioni furono tardive e poco precise, gli azionisti ebbero notizia del motivo per cui Playstation Network era disabilitato solo 7 giorni dopo, e i comunicati ai clienti non furono da meno, alimentando malcontento e sfiducia nelle capacit\`a dell'azienda\cite{Sony_pnt}.
\section{Target (2013)}
\subsection{Descrizione}
Il caso di \textit{data breach} affrontato riguarda l'azienda di distribuzione Target, seconda, negli Stati Uniti, solamente a Walmart, e l'attacco che si \`e concluso con il furto di  informazioni, comprese informazioni di pagamento e carte di credito, di 70 milioni di clienti.\\
Prime avvisaglie dell'ingresso di un malware nella rete arrivano il 30 novembre 2013, probabilmente avvenuto per via dell'utilizzo di passwords troppo semplici nei server aziendali, ma \`e dal 12 dicembre che, in seguito alla notifica del Dipartimento di Giustizia, iniziano le indagini congiunte. Verificatane l'entit\`a, la notizia viene resa di dominio pubblico il 19 dicembre 2013\cite{Target}.\\
\subsection{Impatto diretto}
Da quanto dichiarato nei bilanci SEC (\textit{Securities and Exchange Commission}), le spese dirette, tolti \$90 milioni coperti da assicurazione, ammontano a \$200 milioni\cite{Target}.
\subsection{Impatto indiretto}
Sebbene nei giorni immediatamente successivi ci sia stato un picco verso il basso di -2.2\% \ref{fig:tgt1}, con momenti di ulteriore discesa, e sebbene la crescita sia stata praticamente neutra per un anno, dalla fine del 2014 l'azienda ha recuperato parte della crescita inespressa fino ad allora \ref{fig:tgt2}. Ad aiutare \`e stata l'ottima immagine che ha nei confronti dei clienti e la fiducia dei pi\`u affezionati ad un marchio decisamente forte.\\
\begin{figure}[H] 
\begin{center} 
\includegraphics[width=10cm]{figures/target_short.png} 
\caption[Grafico Target short]{Impatto a breve termine Target data breach}\label{fig:tgt1}
\end{center}
\end{figure}
\begin{figure}[H]
\begin{center} 
\includegraphics[width=10cm]{figures/target_long.png} 
\caption[Grafico Target long]{Impatto a lungo termine Target data breach}\label{fig:tgt2}
\end{center}
\end{figure}

\section{Sony Pictures (2014)}

\subsection{Descrizione}
Sony Pictures Entertainment viene colpito da un attacco informatico di tipo \textit{data breach} a fine novembre 2014, vengono rilasciati 40GB di informazioni che, a detta degli attaccanti, fanno parte di un furto del volume di 100TB. Dentro vi troviamo pi\`u di 6800 dati salariali dei dipendenti, e anticipazioni riguardanti serie TV e film ancora in fase di produzione\cite{SonyPic_buzzfeed}.\\
\subsection{Impatto diretto}
L'azienda dichiara, nel bilancio SEC per aziende estere, di aver speso \$41 milioni per spese investigative e correlate all'attacco e che dovr\`a sostenere spese legali nei confronti di dipendenti ed ex dipendenti\cite{SonyPic_20F_report}. A fine anno trover\`a un accordo per stanziare \$8 milioni di risarcimento.\\

\subsection{Impatto indiretto}
Come notiamo in figura \ref{fig:sPic1}, l'attacco ha progressivamente portato le azioni dell'intero gruppo Sony a -5\% e fino a toccare il -10\% a met\`a dicembre, sopratutto per l'importante risonanza mediatica del caso e il fatto che fosse la seconda volta in pochi anni che il nome della compagnia era accostato a casi di furto di informazioni.\\
\begin{figure}[H] 
\begin{center} 
\includegraphics[width=10cm]{figures/sony_2014_short.png} 
\caption[Grafico Sony Pic short]{Impatto a breve termine Sony Picture hack}\label{fig:sPic1}
\end{center}
\end{figure}

Sul lungo periodo l'andamento si \`e probabilmente svincolato dalla discesa iniziale, anche  grazie alla diffusa gamma di attivit\`a in di cui si occupa il gruppo \ref{fig:sPic2}. L'impatto quindi non \`e quantificabile in maniera semplice, ma \`e ragionevole pensare che il danno di immagine si sia protratto a lungo.\\
\begin{figure}[H] 
\begin{center} 
\includegraphics[width=10cm]{figures/sony_2014_long.png} 
\caption[Grafico Sony PSN long]{Impatto a lungo termine Sony Picture hack}\label{fig:sPic2}
\end{center}
\end{figure}
\subsection{Note}
La copertura dei media e la grande mole di notizie riguardanti il caso derivano dall'associazione dell'attacco alla Corea del Nord e ad un film in uscita, prodotto da SPE, "The Interview", in cui la storia sarebbe stata costruita attorno al tentato assassinio del leader Kim Jong Un.
Nella gestione della crisi, l'essere protagonisti di un evento tanto discusso ha sicuramente contribuito ad aggravare i danni d'immagine connessi alla vicenda\cite{SonyPic_buzzfeed}.\\

\section{Yahoo (2013, 2014)}
\subsection{Descrizione}
A settembre 2016 rende noto che nel 2014 sono stati rubati i dati personali di 500 milioni di utenti a seguito di un attacco di ingegneria sociale\cite{yahoo_book}.\\
Il 14 dicembre 2016 Yahoo comuica che, mentre investigava al riguardo, ha scoperto un secondo furto: durante agosto 2013 \`e avvenuto quello che ricordiamo come il pi\`u grande \textit{data breach} della storia. Inizialmente venne dichiarato si tratasse di 1 miliardo, poi, nel 2017 il dato si assester\`a solidamente sui 3 miliardi di account di cui sono state diffuse informazioni\cite{yahoo_book}\cite{yahoo_guardian}.
\subsection{Impatto diretto}
L'azienda era in fase di acquisizione da parte di Verizon che, ritrattando il prezzo a ribasso, pass\`o da un'offerta di \$4,83 miliardi ad un prezo finale, nel 2017, di \$4,48 determinando una perdita efettiva di \$350 milioni\cite{yahoo_book}.\\
Inoltre i procedimenti portarono l'azienda a pagare altri \$35 milioni per un patteggiamento con la SEC e \$117 milioni per risarcire gli utenti colpiti.\\
\subsection{Impatto indiretto}
Nel breve termine possiamo osservare una discesa di quasi il 10\% da settembre alla fine di dicembre 2016 prima di una crescita che anticipa l'aquisizione.\\
\begin{figure}[H] 
\begin{center} 
\includegraphics[width=10cm]{figures/yahoo_short.png} 
\caption[Grafico Yahoo short]{Impatto a breve termine Yahoo data breach}\label{fig:yahoo1}
\end{center}
\end{figure}

Nel periodo prossimo all'acquisto di Yahoo, Verizon subisce un calo del 6\%. Osservare oltre l'andamento rischia di essere fuorviante per via dei diversi componenti dell'azienda al tempo, ma le azioni seguono un trend in crescita\\
\begin{figure}[H] 
\begin{center} 
\includegraphics[width=10cm]{figures/yahoo-verizon-long.png} 
\caption[Grafico Verizon (Yahoo) long]{Impatto a lungo termine Verizon (Yahoo) data breach}\label{fig:yahoo2}
\end{center}
\end{figure}
\subsection{Note}
\begin{enumerate}
    \item la fonte dei grafici storici di yahoo \`e \href{https://companiesmarketcap.com/yahoo/stock-price-history}{https://companiesmarketcap.com/}, in quanto non \`e stato possibile procurarsi dati pi\`u precisi di una azienda il cui ticker non esiste pi\`u;
    \item la gestione delle comunicazioni da parte dell'azienda \`e stata tardiva a dir poco, contribuendo ad aggravare la posizione dell'azienda nei confronti dell'opinione pubblica e della legge.
\end{enumerate}
\section{Uber (2016, 2022)}
\subsection{2016}
\subsubsection{Descrizione}
Il \textit{data breach} avvenne ad ottobre 2016, ma la divulgazione risalea ben un anno dopo, grazie ad un cambio di dirigenza. I rischi che ha corso nel gestire ambiguamente il caso hanno rischiato di avere conseguenze penali\cite{Uber_plusEquifaxAndYahoo}.\\ 
\subsubsection{Impatto diretto}
L'azienda, oltre al riscatto pagato agli hacker per eliminare i dati rubati, ha concordato coi 50 stati americani il pagamento di una multa di \$148 milioni.\\
\subsubsection{Impatto indiretto}
Al tempo l'azienda non era quotata, quindi non \`e poissibile analizzare questo aspetto, se non osservando il caso del 2022.\\
\subsubsection{Note}
Una aggravante \`e stata sicuramente la gestione consapevolmente ingannevole, per esempio rivel\`o l'esistenza del \textit{breach} a SoftBank, permettendo poi che comprasse una buona parte della compagnia a prezzo ribassato\cite{Uber_plusEquifaxAndYahoo}.\\   
\subsection{2022}
\subsubsection{Descrizione}
Tramite ingegneria sociale l'attaccante \`e riuscito ad entrare nel sistema di condivisione di informazioni aziendali Slack, con conseguente arresto forzato di alcuni servizi da parte dell'azienda\cite{Uber_2022}.\\
\subsubsection{Impatto diretto}
La prontezza della reazione ha permesso di contenere le conseguenze dirette dell'attacco, evitando eventuali sanzioni.\\ 
\subsubsection{Impatto indiretto}
Il mercato ha invecce reagito pi\`u bruscamente di quanto visto fin'ora \ref{fig:ubr1}, con un picco verso il basso di -10\% rispetto al giorno della scoperta (ancor pi\`u pronunciato se prendiamoin esame il 15 settembre, data della dichiarazione). Ad un mese di distanza la discesa rallenta ma non si arresta.\\

\begin{figure}[H] 
\begin{center} 
\includegraphics[width=10cm]{figures/uber_2022_short.png} 
\caption[Grafico Uber 2022 short]{Impatto a breve termine Uber 2022}\label{fig:ubr1}
\end{center}
\end{figure}

Per tornare in pari ci vogliono pi\`u di sette mesi, indice che i precedenti attacchi avevano lasciato un segno importante nella reputazione dell'azienda.\\

\begin{figure}[H] 
\begin{center} 
\includegraphics[width=10cm]{figures/uber_2022_long.png} 
\caption[Grafico Uber 2022 long]{Impatto a lungo termine Uber 2022}\label{fig:ubr2}
\end{center}
\end{figure}

\subsubsection{Note}
Il primo data breach e le sue conseguenze sono servite a Uber per cambiare approccio alla sicurezza e migliorato diversi aspetti\cite{Uber_2022}.

\section{FedEx - NotPetya (2017)}
\subsection{Descrizione}
Il \textit{ransomware} NotPetya fu un attacco diffuso su larga scala, con epicentro in Ucraina, a giugno del 2017.Sua  peculiarit\`a era di non essere stato concepito per scopi di lucro, quanto pi\`u invece per causare disordine e distruzione\cite{FedEx_evolutionOfRansom}, secondo un rapporto della Casa Bianca i danni ammontarono comlessivamente a \$10 miliardi\cite{FedEx_wired}.\\
In questo contesto, l'azienda FedEx Corporation, che aveva da poco fatto acquisizioni in Europa (TNT Express), affront\`o grossi disservizi nel vecchio continente che complicarono i processi di assimilazione\cite{FedEx_10K_report_2018}.\\
\subsection{Impatto diretto}
Nel bilancio depositato a SEC si stimano le perdite intorno ai \$400 milioni che rallentarono la crescita durante il primo quarto dell'anno fiscale (giugno-agosto 2017).\\
\subsection{Impatto indiretto}
Essendo l'azienda distribuita, e in forte crescita negli Stati Uniti, l'impatto distruttivo di NotPetya non \`e roboante come ci si aspetterebbe da uno dei pi\`u costosi attacchi della storia. Di fatti, dal 27 giugno 2017 (giorno dell'attacco) le azioni crescono ed iniziano una discesa solo qualche tempo dopo, scendendo fino a -5\%.\\
\begin{figure}[H] 
\begin{center} 
\includegraphics[width=10cm]{figures/fedex_short.png} 
\caption[Grafico FedEx NotPetya short]{Impatto a breve termine NotPetya su FedEx}\label{fig:fdx1}
\end{center}
\end{figure}

Come gi\`a discusso ci sono segnali di una ripresa abbastanza forte, che in mancanza dell'attacco, sarebbero potuti essere lo specchio di una crescita importante.

\begin{figure}[H] 
\begin{center} 
\includegraphics[width=10cm]{figures/fedex_long.png} 
\caption[Grafico FedEx NotPetya long]{Impatto a lungo termine NotPetya su FedEx}\label{fig:fdx2}
\end{center}
\end{figure}

\subsection{Note}

La diffusione cos\`i ampia di un attacco lo assimila, agli occhi dell'opinione pubblica, ad un disastro naturale o comunque ad un fattore esterno, intaccando pi\`u tramite costi diretti che indiretti le aziende coinvolte. I danni di immagine risultano quindi significativamente contenuti.

\section{Equifax (2017)}
\subsection{Descrizione}
L'8 settembre 2017, l'agenzia di informazioni creditizie al consumo americana Equifax rilascia un comunicato in cui dichiara che ha subito un \textit{data breach} che interessa  148 milioni di cittadini statunitensi. L'attacco \`e stato possibile grazie al mancato aggiornamento tempestivo di alcuni server con una specifica vulnerabilit\`a nota, presente dell'url\cite{Equifax_case_study_2}.\\
\subsection{Impatto diretto}
Come conseguenza della grande quantit\`a di cause indette contro Equifax, l'azienda arriv\`o a concordare un impegno monetario di \$575-\$700 milioni\cite{Equifax_settlement}.\\
\subsection{Impatto indiretto}
L'impatto fu molto pesante, a distanza di tre giorni osserviamo un calo dell' 8,20\%, che va oltre il 20\% a met\`a settembre. Questo \`e dovuto ad una massiccia vedita delle azioni, pari a \$1,8 milioni\cite{Uber_plusEquifaxAndYahoo}\\

\begin{figure}[H] 
\begin{center} 
\includegraphics[width=10cm]{figures/equifax_short.png} 
\caption[Grafico Equifax short]{Impatto a breve termine Equifax breach}\label{fig:eqx1}
\end{center}
\end{figure}

Sul lungo periodo l'azienda fatica a risollevarsi, complici le pesanti sanzioni e la perdita di fiducia degli investitori.\\

\begin{figure}[H] 
\begin{center} 
\includegraphics[width=10cm]{figures/equifax_long.png} 
\caption[Grafico Equifax long]{Impatto a lungo termine Equifax breach}\label{fig:eqx2}
\end{center}
\end{figure}

\subsection{Note}
Pare che il fatto fosse noto alla dirigenza gi\`a da inizio agosto 2017\cite{Uber_plusEquifaxAndYahoo} \`e stata, insieme al tipo di informazioni perse, una pesante aggravante in ambito legale.\\
\section{British Airways (2018)}
\subsection{Descrizione}
Il \textit{data breach}, attuato tramite l'iniezione di 22 linee di codice Javascript malevolo via URL (\textit{cross site scripting}, XSS), che colp\`i la compagnia aerea del Regno Unito, British Airways, venne reso pubblico il 6 settembre 2018 da un comunicato pubblio del \textit{National CyberSecurity Centre}. Le vittime, le cui stime iniziali ne dichiarvano $380\,000$, si scoprono ad ottobre essere ben $565\,000$, di cui sono trapelati nomi, email e informazioni delle carte di credito.\cite{BritAir} 
\subsection{Impatto diretto}
L'azienda ha sostenuto una multa di \$20 milioni come conseguenza dell'attacco.
\subsection{Impatto indiretto}
La compagnia \`e nel gruppo \textit{International Airlines Group} e quotata nelle borse inglesi e spagnole, pertanto discuteremo l'andamento di quest'ultimo tramite i dati ufficiali di \href{https://www.londonstockexchange.com/stock/}{\textit{London Stock Exchange}}.\\
Il gruppo, come evidenziato nelle fonti\cite{BritAir} ha una perdita quasi immediata del 2\%, e continua a scendere durante tutto il mese successivo.

\begin{figure}[H] 
\begin{center} 
\includegraphics[width=10cm]{figures/britAir_short.png} 
\caption[Grafico British Airways short]{Impatto a breve termine British Airways}\label{fig:britair1}
\end{center}
\end{figure}

Nel lungo periodo le azioni risentono del danno di immagine fino all'inizio del 2020, in cui per\`o, la compagnia ha ricadute legate all'inizio della pandemia del Covid-19.\\

\begin{figure}[H] 
\begin{center} 
\includegraphics[width=10cm]{figures/britAir_long.png} 
\caption[Grafico British Airways long]{Impatto a lungo termine  British Airways}\label{fig:fdx2}
\end{center}
\end{figure}

\section{Marriot International (2018)}
\subsection{Descrizione}
L'attacco si distribuisce su un ampio lasso di tempo: nel 2014, tramite accesso fisico ad un terminale con permessi di amministratore, viene installato un \textit{malware} nella rete di Starwood, agenzia alberghiera statunitense. Dopo aver trovato la chiave di decrittazione dei database e aver esfiltrato i dati, gli attaccanti cifrarono nuovamente le informazioni per non lasciare tracce. Cos\`i fu anche dopo l'acquisizione di Starwood da parte di Marriott International nel 2016, tanto da passare inosservata fino all'8 settembre 2018. Seguirono investigazioni che culminarono nella dichiarazione del 30 novembre 2018, in cui si rivelava la diffusione di informazioni di approssimativamente $500$ milioni di clienti in tutto il mondo\cite{Marriott_ResGate}\cite{Marriott_customer_perception}.\\ 
\subsection{Impatto diretto}
Come spese di riparazione interne Marriott dichiara \$30 milioni \cite{Marriott_ResGate}, in aggiunta pag\`o anche una multa di \textsterling18,4 milioni al Regno Unito \cite{Marriott_actual_fine}, sebbene  inizialmente le cifre dichiarate fossero intorno ai \textsterling99 milioni.\\ 
\subsection{Impatto indiretto}

\subsection{Note}
\section{Capital One (2019)}
\subsection{Descrizione}

\subsection{Impatto diretto}

\subsection{Impatto indiretto}

\subsection{Note}
\section{SolarWinds (2020)}
\subsection{Descrizione}
\\
\subsection{Impatto diretto}
filler\\
\subsection{Impatto indiretto}
\\
\subsection{Note}
\\
\section{Colonial Pipeline (2021)}
\subsection{Descrizione}
filler\\
\subsection{Impatto diretto}

\subsection{Impatto indiretto}

\subsection{Note}
\section{MGM Resorts (2023)}
\subsection{Descrizione}

\subsection{Impatto diretto}

\subsection{Impatto indiretto}

\subsection{Note}
\section{Change Healthcare (2024)}
\subsection{Descrizione}

\subsection{Impatto diretto}

\subsection{Impatto indiretto}

\subsection{Note}

\clearpage{\pagestyle{empty}\cleardoublepage}



%%%%%%%%%%%%%%%%%%%%%%%%%%%%%%%%%%%%%%%%
% Analisi dettagliata del precedente capitolo 
\chapter{Analisi dettagliata del precedente capitolo}
Questo sar\`a il capitolo pi\`u importante dove individua trend sugli attacchi e sulle conseguenze.\\
\\href{https://www.overleaf.com/learn/latex/Code\_listing}{listing}.


\clearpage{\pagestyle{empty}\cleardoublepage}



%%%%%%%%%%%%%%%%%%%%%%%%%%%%%%%%%%%%%%%%
% CONCLUSIONI
\chapter{Conclusioni}

Da scrivere alla fine, un breve riassunto di cosa si \`e affrontato, i risultati (se quanto indicato nell'introduzione come obiettivo \`e stato raggiunto) e come \`e possibile continuare la tesi (p.e. se qualcosa non \`e stato affrontato per motivi di tempo o limitazioni hardware).



\clearpage{\pagestyle{empty}\cleardoublepage}



%%%%%%%%%%%%%%%%%%%%%%%%%%%%%%%%%%%%%%%%%%
% RIMUOVERE LE APPENDICI SE NON UTILIZZATE
%imposta l'intestazione di pagina
\renewcommand{\chaptermark}[1]{\markright{\thechapter \ #1}{}}
\lhead[\fancyplain{}{\bfseries\thepage}]{\fancyplain{}{\bfseries\rightmark}}
\appendix %imposta le appendici
\chapter{Prima Appendice} %crea l'appendice
In questa Appendice non si \`e utilizzato il comando:\\
%\verb"" è equivalente all' ambiente verbatim,  ma si utilizza all'interno di un discorso.
\verb"\clearpage{\pagestyle{empty}\cleardoublepage}", ed infatti l'ultima pagina 8 ha l'intestazione con il numero di pagina in alto.
%imposta l'intestazione di pagina
\rhead[\fancyplain{}{\bfseries \thechapter \:Prima Appendice}]
{\fancyplain{}{\bfseries\thepage}}



\clearpage{\pagestyle{empty}\cleardoublepage}



\chapter{Seconda Appendice}
\rhead[\fancyplain{}{\bfseries \thechapter \:Seconda Appendice}]
{\fancyplain{}{\bfseries\thepage}}


\clearpage{\pagestyle{empty}\cleardoublepage}
%%%%%%%%%%%%%%%%%%%%%%%%%%%%%%%%%%%%%%%%%
% BIBLIOGRAFIA
\addcontentsline{toc}{chapter}{Bibliografia}
\label{Bibliography}
\bibliographystyle{IEEEtran}
\bibliography{src/bibliography}
\rhead[\fancyplain{}{\bfseries Bibliografia}]
{\fancyplain{}{\bfseries\thepage}}
\end{document}